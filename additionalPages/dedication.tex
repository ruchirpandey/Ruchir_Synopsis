\begin{center}
{\fontsize{18pt}{20pt}\bookmanheading{\underline{DEDICATION}}}
\thispagestyle{empty}

% ============================================================
%  EPIGRAPH 1 — Composed Shloka (Original, anuṣṭubh meter)
% ============================================================

\begin{center}
{\hindifont\large
सूर्यरश्मिभिराकृष्टं जलं मेघे विलीयते।\\[4pt]
वायुना नीयते दूरं वर्षित्वा सागरं व्रजेत्॥
}
\end{center}

\vspace{0.6cm}

\begin{center}
\textit{Water, drawn up by the sun's rays, dissolves into cloud.}\\[3pt]
\textit{Carried far by the wind, it rains — and returns to the ocean.}
\end{center}

\vspace{0.4cm}
\begin{center}
\textbf{\textit{(Navaracanā — original composition in anuṣṭubh metre)}}
\end{center}

% ============================================================
\begin{center}
\rule{0.35\textwidth}{0.4pt}
\end{center}

% ============================================================
%  EPIGRAPH 2 — Chāndogya Upaniṣad 6.10.1
% ============================================================

\begin{center}
{\hindifont\large
स यथा नद्यः स्यन्दमानाः समुद्रायणाः\\[4pt]
समुद्रं प्राप्यास्तं गच्छन्ति,\\[4pt]
भिद्येते तासां नामरूपे,\\[4pt]
समुद्र इत्येवं प्रोच्यते॥
}
\end{center}

\vspace{0.6cm}

\begin{center}
\textit{Just as rivers, flowing onward, reach the ocean and merge into it}\\[3pt]
\textit{— losing their name and form —}\\[3pt]
\textit{so too are they called simply: the ocean.}
\end{center}

\vspace{0.4cm}
\begin{center}
\textbf{\textit{Chāndogya Upaniṣad 6.10.1}}
\end{center}

% ============================================================
%  DEDICATION
% ============================================================

\vspace{2cm}
\end{center}

\noindent\textbf{Dedicated to,}

\vspace{1.5cm}

\begin{center}
\textit{My parents, wife, lovely kids, teachers and colleagues,}\\[4pt]
\textit{who encouraged and supported me over time.}
\end{center}
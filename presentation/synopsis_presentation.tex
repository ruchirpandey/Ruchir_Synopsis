\documentclass[aspectratio=169,11pt]{beamer}

% Theme and color
\usetheme{Madrid}
\usecolortheme{whale}
\setbeamertemplate{navigation symbols}{}
\setbeamertemplate{footline}[frame number]

% Packages
\usepackage{graphicx}
\usepackage{amsmath}
\usepackage{booktabs}
\usepackage{tikz}
\usepackage{xcolor}

% Custom colors
\definecolor{nitblue}{RGB}{0,51,153}
\definecolor{saffron}{RGB}{255,153,51}
\setbeamercolor{title}{fg=white,bg=nitblue}
\setbeamercolor{frametitle}{fg=white,bg=nitblue}
\setbeamercolor{block title}{fg=white,bg=nitblue}

% Title information
\title[DRL Control for DFIG-Solar PV]{Deep Reinforcement Learning Based Control Strategy for Solar PV-Integrated Doubly-Fed Induction Generator Systems}
\subtitle{Ph.D. Synopsis Presentation}
\author[Ruchir Pandey]{Ruchir Pandey\\[0.3cm]{\small Supervisor: Dr. Sourav Bose}}
\institute[NIT Uttarakhand]{\centering{\includegraphics[width=1.23cm, height=1.07cm]{images/logo.png}}\\[0.1cm]
\centering{Department of Electrical Engineering\\National Institute of Technology, Uttarakhand}}
\date{January 27, 2026}

\begin{document}

% ============== SLIDE 1: Title ==============
\begin{frame}[plain]
\titlepage
\end{frame}

% ============== SLIDE 2: Outline ==============
\begin{frame}{Outline}
\tableofcontents
\end{frame}

% ============== SECTION 1: Introduction ==============
\section{Introduction}

% ============== SLIDE 3: Motivation ==============
\begin{frame}{Motivation}
\begin{columns}
\column{0.55\textwidth}
\textbf{Global Energy Transition:}
\begin{itemize}
    \item Urgent need for decarbonization
    \item Wind and solar PV are dominant renewable sources
    \item Declining costs and technological maturity
\end{itemize}

\vspace{0.3cm}
\textbf{Key Challenges:}
\begin{itemize}
    \item Intermittency and variability
    \item Reduced system inertia
    \item Grid stability concerns
    \item Power quality requirements
\end{itemize}

\column{0.45\textwidth}
\begin{block}{Why Hybrid Systems?}
\begin{itemize}
    \item Complementary generation profiles
    \item Improved power availability
    \item Better resource utilization
    \item Enhanced grid support
\end{itemize}
\end{block}
\end{columns}
\end{frame}

% ============== SLIDE 4: Problem Statement ==============
\begin{frame}{Problem Statement}
\begin{alertblock}{Control Challenges in DFIG-PV Hybrid Systems}
Integration of solar PV with DFIG wind turbines introduces:
\begin{itemize}
    \item \textbf{Tightly coupled nonlinear dynamics} between RSC, GSC, and PV
    \item \textbf{Multi-timescale disturbances} (wind: seconds, solar: sub-seconds)
    \item \textbf{DC Link Voltage Control}
\end{itemize}
\end{alertblock}

\vspace{0.3cm}
\begin{block}{Limitations of Conventional PI Control}
\begin{itemize}
    \item Requires accurate system models
    \item Extensive parameter tuning needed
    \item Performance degrades under varying conditions
    \item Cannot handle multi-objective optimization
\end{itemize}
\end{block}
\end{frame}

% ============== SLIDE 5: Research Gap ==============
\begin{frame}{Research Gap}
\begin{columns}
\column{0.5\textwidth}
\textbf{Existing DRL Applications:}
\begin{itemize}
    \item TD3 for PMSG wind turbines
    \item DDPG for microgrid control
    \item DRL for wind farm frequency regulation
    \item Virtual inertia control
\end{itemize}

\column{0.5\textwidth}
\begin{alertblock}{Identified Gap}
Comprehensive study aimed at application of \textbf{DDPG} and \textbf{TD3}  to integrated \textbf{DFIG-Solar PV} hybrid systems with:
\begin{itemize}
    \item Direct DC-link coupling
    \item Unified RSC-GSC-PV control
    \item HIL experimental validation
\end{itemize}
\end{alertblock}
\end{columns}
\end{frame}

% ============== SECTION 2: Objectives ==============
\section{Research Objectives}

% ============== SLIDE 6: Objectives ==============
\begin{frame}{Research Objectives}
\textbf{Primary Objective:}\\
Develop, implement, and experimentally validate DRL-based control strategies (DDPG and TD3) for solar PV-integrated DFIG wind energy systems.

\vspace{0.4cm}
\textbf{Specific Objectives:}
\begin{enumerate}
    \item \textbf{Algorithm Implementation:}
    \begin{itemize}
        \item Implement DDPG and TD3 for converter control
    \end{itemize}

    \item \textbf{Performance Validation:}
    \begin{itemize}
        \item Conduct HIL experiments on OPAL-RT platform
        \item Compare DRL with classical PI control
    \end{itemize}

    \item \textbf{Critical Analysis:}
    \begin{itemize}
        \item Comparative analysis of DDPG vs. TD3
    \end{itemize}
\end{enumerate}
\end{frame}

% ============== SECTION 3: System Configuration ==============
\section{System Configuration}

% ============== SLIDE 7: System Architecture ==============
\begin{frame}{DFIG-Solar PV Hybrid System Architecture}
\begin{columns}
\column{0.45\textwidth}
\textbf{Key Components:}
\begin{itemize}
    \item DFIG wind turbine
    \item Rotor Side Converter (RSC)
    \item Grid Side Converter (GSC)
    \item Solar PV array at DC link
    \item DC link capacitor
\end{itemize}

\vspace{0.2cm}
\textbf{Control Objectives:}
\begin{itemize}
    \item \textbf{RSC:} Rotor current, torque, MPPT
    \item \textbf{GSC:} DC-link voltage, grid power
    \item \textbf{PV:} Maximum power extraction
\end{itemize}

\column{0.55\textwidth}
\begin{figure}
    \centering
    \includegraphics[width=\textwidth]{images/Updated_DFIG_Topology.png}
    \caption{DFIG-Solar PV hybrid system topology}
\end{figure}
\end{columns}
\end{frame}

% ============== SECTION 4: Methodology ==============
\section{Methodology}

% ============== SLIDE 8: DRL Background ==============
\begin{frame}{Deep Reinforcement Learning for Power Systems}
\begin{columns}
\column{0.55\textwidth}
\textbf{Why DRL?}
\begin{itemize}
    \item Model-free learning
    \item Handles nonlinear, coupled dynamics
    \item Continuous action spaces
    \item Adaptive behavior
    \item No explicit system model required
\end{itemize}

\vspace{0.3cm}
\textbf{Actor-Critic Framework:}
\begin{itemize}
    \item \textbf{Actor:} Learns policy $\pi(s) \rightarrow a$
    \item \textbf{Critic:} Evaluates Q-value $Q(s,a)$
\end{itemize}

\column{0.45\textwidth}
\begin{block}{RL Formulation}
\begin{itemize}
    \item \textbf{State:} $V_{dc}$, $i_r$, $\omega_r$, $P_{pv}$
    \item \textbf{Action:} Converter control signals
    \item \textbf{Reward:} Multi-objective function
\end{itemize}
\end{block}

\begin{equation*}
r = -\alpha|V_{dc} - V_{ref}| - \beta|P - P_{ref}|
\end{equation*}
\end{columns}
\end{frame}

% ============== SLIDE 9: DDPG Algorithm ==============
\begin{frame}{Deep Deterministic Policy Gradient (DDPG)}
\begin{columns}
\column{0.5\textwidth}
\textbf{Key Features:}
\begin{itemize}
    \item Off-policy actor-critic algorithm
    \item Continuous action spaces
    \item Experience replay buffer
    \item Target networks for stability
    \item Deterministic policy gradient
\end{itemize}

\vspace{0.3cm}
\textbf{Critic Update:}
\begin{equation*}
L = \mathbb{E}[(Q(s,a) - y)^2]
\end{equation*}
\begin{equation*}
y = r + \gamma Q'(s', \pi'(s'))
\end{equation*}

\column{0.5\textwidth}
\begin{alertblock}{DDPG Limitations}
\begin{itemize}
    \item Q-value overestimation bias
    \item Training instability
    \item Sensitivity to hyperparameters
    \item Aggressive policy updates
\end{itemize}
\end{alertblock}

\vspace{0.3cm}
These limitations motivate the development of \textbf{TD3} algorithm.
\end{columns}
\end{frame}

% ============== SLIDE 10: TD3 Algorithm ==============
\begin{frame}{Twin-Delayed Deep Deterministic Policy Gradient (TD3)}
\textbf{TD3 Innovations over DDPG:}

\begin{columns}
\column{0.33\textwidth}
\begin{block}{1. Clipped Double Q-Learning}
\begin{itemize}
    \item Twin critic networks
    \item Takes minimum Q-value
    \item Mitigates overestimation
\end{itemize}
\begin{equation*}
y = r + \gamma \min_{i=1,2} Q_i'
\end{equation*}
\end{block}

\column{0.33\textwidth}
\begin{block}{2. Delayed Policy Updates}
\begin{itemize}
    \item Update policy less frequently
    \item Typically every 2 critic updates
    \item Enhanced stability
\end{itemize}
\end{block}

\column{0.33\textwidth}
\begin{block}{3. Target Policy Smoothing}
\begin{itemize}
    \item Add noise to target actions
    \item Regularization effect
    \item Improved robustness
\end{itemize}
\begin{equation*}
\tilde{a} = \pi'(s') + \epsilon
\end{equation*}
\end{block}
\end{columns}

\vspace{0.3cm}
\begin{alertblock}{Result}
Longer training sessions but improved transient response 
\end{alertblock}
\end{frame}

% ============== SECTION 5: Experimental Setup ==============
\section{Experimental Validation}

% ============== SLIDE 11: HIL Setup ==============
\begin{frame}{Hardware-in-the-Loop Experimental Setup}
\begin{columns}
\column{0.5\textwidth}
\textbf{OPAL-RT OP5700 Platform:}
\begin{itemize}
    \item Real-time digital simulator
    \item Intel Xeon processor, RedHawk Linux RTOS
    \item 1 ms sampling time
\end{itemize}

\vspace{0.2cm}
\textbf{Test Scenarios:}
\begin{itemize}
    \item Steady-state operation
    \item Wind speed variations (10--11.5 m/s)
    \item Solar irradiance steps (0--5 A)
    \item Combined disturbances
\end{itemize}

\column{0.5\textwidth}
\begin{figure}
    \centering
    \includegraphics[width=0.9\textwidth]{images/opalrt.png}
    \caption{OPAL-RT HIL platform}
\end{figure}
\end{columns}
\end{frame}

% ============== SLIDE: Experimental Runs ==============
\begin{frame}{Experimental Runs Overview}
\begin{columns}
\column{0.55\textwidth}
\begin{block}{Experimental Runs}
\begin{tabular}{|l|l|}
\hline
\textbf{Run} & \textbf{Objective} \\
\hline
Run 1 & DFIG baseline (no PV) \\
Run 2 & With solar integration \\
Run 3 & Comparative analysis \\
Run 4 & Multi-scenario tests \\
Run 5 & Combined disturbances \\
\hline
\end{tabular}
\end{block}

\vspace{0.3cm}
\textbf{Controllers Tested:}
\begin{itemize}
    \item Conventional PI control
    \item DDPG-based control
    \item TD3-based control
\end{itemize}

\column{0.45\textwidth}
\textbf{Measured Parameters:}
\begin{itemize}
    \item DC link voltage ($V_{dc}$)
    \item Rotor current ($i_r$)
    \item Solar power ($P_{solar}$)
    \item Grid power ($P_{grid}$)
    \item Rotor power ($P_{rotor}$)
\end{itemize}

\vspace{0.3cm}
\textbf{Performance Metrics:}
\begin{itemize}
    \item Rise time, settling time
    \item Overshoot percentage
    \item Voltage regulation
\end{itemize}
\end{columns}
\end{frame}

% ============== SECTION 6: Results ==============
\section{Results and Analysis}

% ============== SLIDE 12: DC Link Voltage Results ==============
\begin{frame}{DC Link Voltage Response}
\begin{columns}
\column{0.5\textwidth}
\textbf{Wind Speed Step (10 $\rightarrow$ 11.2 m/s):}

\vspace{0.3cm}
\begin{tabular}{lccc}
\toprule
\textbf{Metric} & \textbf{TD3} & \textbf{DDPG} & \textbf{PI} \\
\midrule
Peak Deviation & 12\% & 15\% & 18\% \\
Rise Time (ms) & 12 & 13 & 15 \\
Settling (ms) & 98 & 102 & 118 \\
\bottomrule
\end{tabular}

\vspace{0.3cm}
\textbf{Key Findings:}
\begin{itemize}
    \item TD3: Fastest recovery
    \item DDPG: Minor oscillations
    \item PI: Largest overshoot
\end{itemize}

\column{0.5\textwidth}
\textbf{Solar PV Step (0 $\rightarrow$ 5 A):}

\vspace{0.3cm}
\begin{tabular}{lccc}
\toprule
\textbf{Controller} & \textbf{Regulation} & \textbf{Settling} \\
\midrule
TD3 & $\pm$4.6\% & $<$ 90 ms \\
DDPG & $\pm$4.8\% & $\sim$ 100 ms \\
PI & $\pm$5.0\% & $>$ 120 ms \\
\bottomrule
\end{tabular}

\vspace{0.3cm}
TD3 maintains tightest voltage regulation under solar disturbances
\end{columns}
\end{frame}

% ============== SLIDE 13: DC Link Voltage Experimental Results ==============
\begin{frame}{DC Link Voltage: Experimental Results}
\begin{columns}
\column{0.33\textwidth}
\begin{figure}
    \centering
    \includegraphics[width=\textwidth]{images/PID_Vdc.png}
    \caption{PI Controller}
\end{figure}

\column{0.33\textwidth}
\begin{figure}
    \centering
    \includegraphics[width=\textwidth]{images/DDPG_Vdc.png}
    \caption{DDPG Controller}
\end{figure}

\column{0.33\textwidth}
\begin{figure}
    \centering
    \includegraphics[width=\textwidth]{images/TD3_Vdc.jpg}
    \caption{TD3 Controller}
\end{figure}
\end{columns}

\vspace{0.3cm}
\begin{center}
\textbf{TD3 demonstrates fastest recovery and tightest voltage regulation}
\end{center}
\end{frame}

% ============== SLIDE 14: Rotor Current Results ==============
\begin{frame}{Rotor Current Dynamic Response}
\textbf{Simultaneous Wind + Solar Variations (Most Challenging Scenario):}

\begin{columns}
\column{0.5\textwidth}
\begin{block}{Performance Comparison}
\begin{tabular}{lccc}
\toprule
\textbf{Metric} & \textbf{TD3} & \textbf{DDPG} & \textbf{PI} \\
\midrule
Rise Time (ms) & 12 & 13 & 15 \\
Settling (ms) & 34 & 36 & 40 \\
Overshoot (\%) & 4.4 & 4.6 & 5.0 \\
\bottomrule
\end{tabular}
\end{block}

\vspace{0.2cm}
\textbf{TD3 Advantages:}
\begin{itemize}
    \item Smoothest transient response
    \item Optimal damping
    \item No secondary oscillations
\end{itemize}

\column{0.5\textwidth}
\textbf{Improvements:}

\vspace{0.3cm}
\textbf{TD3 vs PI:}
\begin{itemize}
    \item 20\% faster rise time
    \item 15\% faster settling
    \item 12\% lower overshoot
\end{itemize}

\vspace{0.3cm}
\textbf{TD3 vs DDPG:}
\begin{itemize}
    \item 7.7\% faster rise time
    \item 5.6\% faster settling
    \item 4.3\% lower overshoot
\end{itemize}
\end{columns}
\end{frame}

% ============== SLIDE 15: Rotor Current Experimental Results ==============
\begin{frame}{Rotor Current: Experimental Results}
\begin{columns}
\column{0.33\textwidth}
\begin{figure}
    \centering
    \includegraphics[width=\textwidth]{images/PID_Rotor_current.png}
    \caption{PI Controller}
\end{figure}

\column{0.33\textwidth}
\begin{figure}
    \centering
    \includegraphics[width=\textwidth]{images/DDPG_Rotor_current.png}
    \caption{DDPG Controller}
\end{figure}

\column{0.33\textwidth}
\begin{figure}
    \centering
    \includegraphics[width=\textwidth]{images/TD3_Rotor_current.png}
    \caption{TD3 Controller}
\end{figure}
\end{columns}

\vspace{0.3cm}
\begin{center}
\textbf{TD3 shows smoothest transient response with optimal damping}
\end{center}
\end{frame}

% ============== SLIDE 16: Overall Performance ==============
\begin{frame}{Overall System Performance Comparison}
\begin{center}
\begin{tabular}{lccc}
\toprule
\textbf{Metric} & \textbf{TD3} & \textbf{DDPG} & \textbf{PI Control} \\
\midrule
Response Time (ms) & \textbf{72} & 80 & 85 \\
Power Overshoot (\%) & \textbf{7.0} & 7.2 & 7.8 \\
DC Link Voltage Regulation & \textbf{$\pm$4.6\%} & $\pm$4.8\% & $\pm$5\% \\
Settling Time (ms) & \textbf{98} & 102 & 118 \\
Power Factor & 0.97--0.99 & 0.97--0.99 & 0.95--0.98 \\
\bottomrule
\end{tabular}
\end{center}

\vspace{0.3cm}
\begin{columns}
\column{0.5\textwidth}
\begin{block}{TD3 vs PI Control}
\begin{itemize}
    \item 15.3\% faster response
    \item 10.3\% lower overshoot
    \item 16.9\% faster settling
\end{itemize}
\end{block}

\column{0.5\textwidth}
\begin{block}{TD3 vs DDPG}
\begin{itemize}
    \item 10\% faster response
    \item 4.2\% tighter regulation
    \item More stable training
\end{itemize}
\end{block}
\end{columns}
\end{frame}

% ============== SLIDE 15: Solar Integration Benefits ==============
\begin{frame}{Solar PV Integration Benefits}
\begin{columns}


\column{0.5\textwidth}
\textbf{DC Link Voltage Improvement:}

\vspace{0.3cm}
\begin{tabular}{|l|c|c|}
\hline
\textbf{Controller} & \textbf{Run 1} & \textbf{Run 2} \\
\hline
PI & $\pm$5.0\% & $\pm$4.2\% \\
DDPG & $\pm$4.8\% & $\pm$4.0\% \\
TD3 & $\pm$4.6\% & $\pm$3.8\% \\
\hline
\end{tabular}

\vspace{0.3cm}
Solar PV contributes to DC link stability (16-17\% improvement).
\end{columns}
\end{frame}

% ============== SLIDE: Solar Power Integration Results ==============
\begin{frame}{Solar PV Integration: Experimental Validation}
\begin{columns}
\column{0.5\textwidth}
\begin{figure}
    \centering
    \includegraphics[width=\textwidth]{images/Run2_M1_PSolar.png}
    \caption{Solar power response (Run 2)}
\end{figure}

\column{0.5\textwidth}
\begin{figure}
    \centering
    \includegraphics[width=\textwidth]{images/Run2_M2_Powers.png}
    \caption{System power response (Run 2)}
\end{figure}
\end{columns}

\vspace{0.2cm}
\begin{center}
\textbf{Successful solar PV integration with smooth power transitions}
\end{center}
\end{frame}

% ============== SECTION 7: Contributions ==============
\section{Contributions}

% ============== SLIDE 16: Major Contributions ==============
\begin{frame}{Major Contributions}
\begin{enumerate}
    \item \textbf{DRL Application to DFIG-PV Hybrid Systems}
    \begin{itemize}
        \item Application of DDPG and TD3 to integrated solar PV-DFIG systems
    \end{itemize}

    \vspace{0.2cm}
    \item \textbf{Comprehensive DDPG-TD3 Comparison}
    \begin{itemize}
        \item Identical test conditions for fair evaluation
        \item Quantified performance improvements
    \end{itemize}

    \vspace{0.2cm}
    \item \textbf{Unified Controller Design}
    \begin{itemize}
        \item Single DRL framework for RSC, GSC, and PV coordination
        \item Multi-objective reward function
    \end{itemize}

    \vspace{0.2cm}
    \item \textbf{HIL Experimental Validation}
    \begin{itemize}
        \item OPAL-RT OP4510 real-time validation
       
    \end{itemize}
\end{enumerate}
\end{frame}

% ============== SLIDE 17: Publications ==============


% ============== SECTION 8: Conclusion ==============
\section{Conclusion}

% ============== SLIDE 18: Conclusion ==============
\begin{frame}{Conclusion}
\begin{block}{Key Findings}
\begin{itemize}
    \item TD3 achieves \textbf{15\% faster response} and \textbf{17\% faster settling} than PI
    \item Both DRL controllers outperform classical PI control
    \item Solar PV integration improves DC-link stability 
    %
    \item HIL validation confirms practical  viability
\end{itemize}
\end{block}

\vspace{0.3cm}
\begin{block}{Trade-offs}
\begin{itemize}
    \item TD3 has higher training complexity but better dynamic performance
    \item DDPG shows aggressive optimization, TD3 offers balanced multi-objective performance
    \item Model-free nature enables adaptive control without accurate system models
\end{itemize}
\end{block}
\end{frame}
\begin{frame}{Publications}
\textbf{Journal Publications:}
\begin{enumerate}
    \item R. Pandey, S. Bose, and P. Dwivedi, ``DDPG algorithm for power optimization and control of solar PV-integrated DFIG wind energy systems,'' \textit{Scientific Reports}, vol. 15, Article 39212, 2025. \textbf{(Published)}

    \vspace{0.3cm}
    \item R. Pandey et al., ``Twin-delayed deep deterministic policy gradient for enhanced power optimization in solar PV-integrated DFIG wind energy systems,'' \textit{Scientific Reports}, 2025. \textbf{(Under Review)}
\end{enumerate}

\vspace{0.4cm}
\textbf{Conference Publication:}
\begin{enumerate}
    \item R. Pandey, S. Bose, P. Dwivedi, and S. Negi, ``Real-time performance investigation of a solar PV integrated DFIG system,'' in \textit{Proc. IEEE PIICON}, New Delhi, India, Nov. 2022.
\end{enumerate}
\end{frame}
% ============== SLIDE 19: Future Scope ==============
\begin{frame}{Future Scope}


\textbf{Scalability and Transferability Approaches:} 
\begin{itemize}
    \item Future work should investigate Multi-Agent Reinforcement Learning (MARL) frameworks, where each converter or turbine acts as a cooperative agent, to enable decentralized control of entire wind and solar farms 
\end{itemize}



\textbf{Advanced Comparative Studies and Hybrid Architectures:}
\begin{itemize}
    \item  An exhaustive comparative analysis should be conducted against other state-of-the-art control strategies, including Model Predictive Control (MPC) and other DRL algorithms like Soft Actor-Critic (SAC) 
\end{itemize}



\end{frame}

% ============== SLIDE 20: Thank You ==============
\begin{frame}[plain]
\begin{center}
\vspace{2cm}
{\Huge\textbf{Thank You}}




\end{center}
\end{frame}

\end{document}

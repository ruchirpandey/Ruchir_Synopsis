\section{Thesis Organization}

The thesis is structured to provide a comprehensive treatment of DRL-based control for DFIG-PV hybrid systems, progressing from theoretical foundations through experimental validation to critical analysis:

\begin{itemize}
    \item \textbf{Chapter~\ref{chap:introduction}}  establishes the research context, motivation, and objectives, including recent advances in DRL applications for renewable energy systems.
    
    \item \textbf{Chapter~\ref{chap:literature}} provides an extensive survey of relevant literature spanning: hybrid renewable energy systems, DFIG control strategies, solar PV integration techniques, deep reinforcement learning fundamentals, and recent applications of DRL in power systems with emphasis on 2023-2025 developments.
    
    \item \textbf{Chapter A.3} details the comprehensive system modeling framework including: DFIG electromagnetic and mechanical models, solar PV single-diode equivalent circuit, DC-link dynamics, power converter models (RSC and GSC).
    
    \item \textbf{Chapter A.4} introduces Control strategies used, fundamentals of deep reinforcement learning including policy gradient methods, actor-critic architectures, and specific formulations of DDPG and TD3 algorithms with emphasis on power system applications.
    
    \item \textbf{Chapter A.5} describes DDPG methodology including: environment formulation, state-action space design, reward function development, neural network architectures, training procedures, and preliminary simulation results.
    
    \item \textbf{Chapter A .6} presents the TD3 algorithm including: algorithmic enhancements over DDPG, implementation details, hyperparameter optimization, curriculum learning strategies, and comparative performance analysis. It also outlines the comprehensive HIL experimental setup including: OPAL-RT platform specifications, system parameter configuration, test scenario design \cite{Prabakar2019_NREL,vonJouanne2023}.    
  
    
        
   
\end{itemize}

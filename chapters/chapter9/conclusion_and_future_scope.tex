% \chapter{RESULT DISCUSSION AND CONCLUSION}
% \label{ch:conclusion}

\section{Introduction}
The integration of solar photovoltaic (PV) arrays into the DC link of Doubly-Fed Induction Generator (DFIG) wind turbines presents a highly efficient hybrid architecture that reduces component count and minimizes cascading conversion losses. However, as established in Chapter 1 and detailed mathematically in Chapter 3, this topology introduces severe control complexities. The system is characterized by tightly coupled, nonlinear multi-timescale dynamics wherein the Rotor-Side Converter (RSC), Grid-Side Converter (GSC), and the intermittent solar PV array continuously interact. Classical control paradigms, primarily dependent on cascaded Proportional-Integral (PI) loops, often struggle to manage these highly coupled interactions effectively, typically resulting in suboptimal transient responses and compromised DC link voltage regulation.

This thesis systematically investigated, implemented, and validated Deep Reinforcement Learning (DRL) algorithms---specifically the Deep Deterministic Policy Gradient (DDPG) and the Twin-Delayed Deep Deterministic Policy Gradient (TD3)---to formulate a unified, adaptive control strategy. By mapping an 11-dimensional state space directly to a 4-dimensional continuous action space (the d-q axis voltage references for the RSC and GSC), the proposed DRL framework completely replaced the conventional architecture of multiple independent PI controllers. This chapter provides a comprehensive analysis and discussion of the empirical findings obtained from the OPAL-RT OP5700 Hardware-in-the-Loop (HIL) experimental validations presented in Chapter 5. It draws deep connections to the theoretical foundations discussed in Chapters 2 and 4, elucidating why the DRL algorithms, and TD3 in particular, significantly outperformed classical methods. 

\section{Comprehensive Discussion of Experimental Findings}

\subsection{Effectiveness of the Unified Control Architecture}
The experimental validation confirmed the profound operational advantages of the proposed unified control framework. In classical control schemes, the RSC manages active and reactive power (or torque and rotor flux) independently of the GSC, which focuses solely on DC link voltage and grid-side reactive power. However, when a 500~W solar PV array is integrated directly into the 230~V DC link, any sudden variation in solar irradiance behaves as an instantaneous disturbance to the DC link power balance equation: 
\begin{equation}
    C \frac{dv_{dc}}{dt} = i_{pv} + i_{r,dc} - i_{g,dc}
\end{equation}

The DRL agents successfully managed this complex power balance without relying on explicitly decoupled control loops. By perceiving the entire system state---including DFIG electrical states, DC link voltage, PV current, and grid power flow---the single neural network actor inherently learned the cross-coupling dynamics. The empirical results demonstrated that solar power injection (such as the 0~A to 5~A step response) caused a proportional reduction in grid power import/export requirements while leaving the rotor-side Maximum Power Point Tracking (MPPT) completely undisturbed. This confirms that the DRL reward function, which simultaneously penalized frequency deviations, active/reactive power tracking errors, and DC link voltage deviations, successfully forced the neural network to discover a globally optimal control manifold that manages all subsystems concurrently.

\subsection{Comparative Transient and Steady-State Performance}
The comparative analysis between TD3, DDPG, and the PI baseline across diverse operational scenarios (simultaneous step changes, realistic profiles, and combined disturbances) highlights the strict superiority of learning-based adaptive control over fixed-gain linear methods.

\begin{itemize}
    \item \textbf{Response and Settling Times:} The TD3 controller achieved the fastest overall system response time of 72~ms, outperforming DDPG (80~ms) and significantly surpassing the PI controller (85~ms). In terms of system settling time, TD3 reached steady-state within 98~ms, compared to 102~ms for DDPG and 118~ms for PI. In the specific context of the RSC, TD3 reduced the rise time to a mere 12~ms and the settling time to 34~ms (compared to the PI controller's 15~ms and 40~ms, respectively). This rapid dynamic response is critical in modern low-inertia grids, as it enables the wind-solar hybrid system to provide faster frequency support and effectively ride through rapid wind gusts or cloud transients.
    
    \item \textbf{Power Overshoot:} The minimization of power overshoot is vital for preventing transient mechanical stresses on the turbine shaft and avoiding electrical saturation in the converters. The TD3 algorithm recorded an overall power overshoot of 7.0\%, while DDPG recorded 7.2\% and PI reached 7.8\%. The PI controller's higher overshoot is a fundamental consequence of integral windup and fixed gains that cannot optimally adapt to large-signal nonlinear disturbances. 
    
    \item \textbf{DC Link Voltage Regulation:} Maintaining the DC link voltage precisely at the 230~V reference is the most critical metric for the GSC in a PV-integrated system. The classical PI controller maintained regulation within $\pm 5\%$, sitting exactly at the threshold of acceptable grid code compliance. The DDPG controller improved this to $\pm 4.8\%$. However, TD3 achieved the tightest regulation at $\pm 4.6\%$. During severe simultaneous disturbances (e.g., wind speed stepping from 10~m/s to 11.2~m/s while solar current ramps to 5~A), TD3 maintained the DC link voltage remarkably flat with virtually zero steady-state error, proving its superior disturbance rejection capabilities.    
   
\end{itemize}

\subsection{Analysis of Subsynchronous and Supersynchronous Operation}
A notable achievement observed during the extensive testing (Runs 1 through 5) was the seamless transition between operating modes. At a wind speed of 10.0~m/s (subsynchronous mode), mechanical power extraction resulted in the grid supplying power to the DC link to feed the rotor. The DRL controllers seamlessly recognized the reversal of the $i_{r,dc}$ power flow and compensated by utilizing the solar PV power to offset the grid import requirement. When wind speed increased to 11.5~m/s (supersynchronous mode), the power flow reversed naturally, and the unified TD3 controller maintained perfectly balanced sinusoidal rotor currents without any degradation in MPPT tracking or grid power quality.

\section{Theoretical Insights and Algorithmic Merits}
The quantitative performance differences observed in the OPAL-RT HIL simulations can be directly mapped to the algorithmic architecture of the respective control strategies detailed in Chapters 2 and 4.

\subsection{Overcoming DDPG's Overestimation Bias}
While DDPG provided a robust baseline that outperformed PI control, it exhibited minor secondary oscillations during severe transients, settling at 36~ms for RSC responses versus TD3's 34~ms. This behavior is theoretically linked to DDPG's single-critic architecture. In Q-learning-based continuous control, approximation errors in the neural network critic lead to an overestimation bias of the action-value function: 
\begin{equation}
    \mathbb{E}[Q_{\phi}(s,a)] > Q^{*}(s,a)
\end{equation}
Because the actor updates its policy by ascending the Q-value gradient, it inadvertently exploits these overestimated peaks, leading to overly aggressive, high-variance control voltage references. In the physical DFIG-PV system, this aggressive policy execution manifested as slightly higher transient overshoots (7.2\% overall) and minute oscillatory behaviors as the controller continuously ``over-corrected'' the power converters.

\subsection{The Efficacy of Clipped Double Q-Learning}
The TD3 algorithm explicitly neutralized the aforementioned vulnerability. By employing two structurally identical but independently initialized critic networks and utilizing the minimum of the two for target value computation:
\begin{equation}
    y = r + \gamma \min_{i=1,2} Q_{\phi_i'}(s', \tilde{a}')
\end{equation}
TD3 forced a pessimistic, conservative evaluation of the state-action space. This theoretical modification directly explains why the TD3-controlled hybrid system exhibited zero secondary oscillations and the lowest overshoot metrics (4.4\% in the RSC). The conservative value estimation translated into smoother, safer voltage commands to the physical switching logic, reducing electrical stress on the IGBTs and allowing for tighter integration of the intermittent PV power.

\subsection{Target Policy Smoothing and Generalization}
A key requirement for any power system controller is robustness against measurement noise and fluctuating environmental conditions. The TD3 algorithm implemented target policy smoothing by injecting clipped Gaussian noise ($\sigma=0.2$, $c=0.5$) into the target actions during critic training. This acts as a regularizer, forcing the critics to evaluate a localized area around the intended action rather than a singular, potentially brittle peak. The practical result observed in the HIL results was TD3's exceptional disturbance rejection. When subjected to highly variable, continuous NREL solar irradiance profiles and realistic Van der Hoven spectrum wind profiles, TD3 maintained strict adherence to the references without exhibiting chaotic switching patterns, proving that the policy had generalized beautifully across the state space rather than overfitting to specific training transients.

\subsection{Stabilized Learning via Delayed Policy Updates}
The training convergence profiles discussed in Chapter 5 revealed that TD3 required 2500 episodes compared to DDPG's 2000 episodes to achieve full convergence. However, TD3 achieved a 40\% lower variance in episode rewards. This stability is attributed to the delayed policy update mechanism ($d=2$). By allowing the twin critics to update twice before shifting the actor network, TD3 decoupled the policy parameters from the high-variance fluctuations of the temporal difference error. This ensured that the actor network only updated against a highly accurate, converged value landscape. The curriculum learning approach---staging from purely DC-link regulation to full 6-objective optimization---further synergized with this delay, allowing the complex 125,504-parameter actor network to discover the global optimum safely.

\section{Practical Implementation and Computational Trade-offs}
Moving advanced DRL methodologies from simulation to real-time industrial deployment requires a rigorous assessment of computational constraints. The analysis of the algorithms provides critical insights into their real-world feasibility.

\subsection{Training Overhead vs. Deployed Efficiency}
The implementation of TD3 came with a measurable computational penalty during the offline training phase. The utilization of twin critics (doubling the critic parameter count to 253,402) and the necessity of running 2500 longer episodes resulted in a training time of approximately 12 hours on an NVIDIA T4 GPU, representing a 50\% increase over DDPG's 8-hour training time. 

However, a fundamental conclusion of this thesis is that this training overhead is strictly a one-time offline cost. During online real-time deployment (inference phase), the critic networks and the experience replay buffers are entirely discarded. The only active component is the deterministic actor network. Both DDPG and TD3 utilize the identical $[11]-[400]-[300]-[4]$ architecture for the actor. Consequently, their inference complexity is identical: $\mathcal{O}(125,504)$ operations per timestep.

\subsection{Hardware-in-the-Loop Real-Time Viability}
The experimental deployment on the OPAL-RT OP4500 real-time simulator shows much greater performance than classical PI controller. Although only the voltage  and current control were implemented while not including  protection and fault conditions, the controller shows good results. The study under fault conditions and protection circuit need to be carried out to implement in real hardware scenerio for safety. The trained model was able to work satisfactorily with the OPAL-RT simulator.The DDPG based learning took less time to train than TD3 but TD3 thus provides better control performance than DDPG at zero additional real-time computational cost.

\section{Conclusions}

\subsection{Fulfillment of Research Objectives}
The primary objective of this thesis---to develop, implement, and experimentally validate DRL-based control strategies for hybrid DFIG-Solar PV energy systems---has been  achieved. The research successfully transitioned from the theoretical formulation of the tightly coupled, nonlinear DFIG and PV single-diode mathematical models to the practical realization of a unified controller. The OPAL-RT validation confirmed that the DRL algorithms successfully learned to manage the complex environment, showing better results than classical PI control in both steady-state and transient conditions.

\subsection{Novel Contributions}
This research makes several distinct contributions to the fields of power electronics and artificial intelligence:
\begin{enumerate}
    \item \textbf{ Implementation of TD3 in DFIG-PV Hybrid Systems:} This work successfully closed the literature gap by applying the advanced TD3 algorithm to a direct DC-link coupled wind-solar hybrid system.
    \item \textbf{Unified Control Framework:} The thesis demonstrated that a single actor network could successfully replace up to eight independent PI control loops, implicitly managing the complex cross-coupling between rotor currents, DC-link voltages, and grid power injection.
    \item \textbf{DDPG and TD3 Algorithmic Comparison:}  DDPG and TD3 were applied in same problem  using similar environment, parameters, state/action spaces, and constraints.The comparison proved that TD3 improved performance over the DDPG algorithm.
    \item \textbf{HIL Validation of Real-Time DRL:} The study used the trained weights from RL model learning from google colab environment. These weights were used in simulink model for OPAL-RT real time simulation.  
\end{enumerate}

\section{Possibilities for Hardware Deployment}
Based on the detailed empirical data generated in this study, the following suggestions are provided for hardware deployment as DDPG scheme serves as a highly efficient tool for applications that prioritize rapid prototyping. The TD3 could be more suitable for safety critical grid applications. The increased training time is a acceptable given the benefits with the improved DC-link regulation,response time and improved overshoot offered by twin-critic architecture.
   

\section{Limitations of the Present Study}
While the findings are robust, certain limitations must be acknowledged. First, deep neural networks inherently lack formal Lyapunov stability proofs, making theoretical verification of global stability impossible. Second, the controllers were trained and validated on a specific 7.5~kW DFIG and 500~W PV scale; direct transferability to multi-megawatt offshore turbines without retraining has not been verified, as physical time constants scale differently with size. Finally, while OPAL-RT HIL is the industry standard for validation, it cannot perfectly replicate long-term component degradation, such as IGBT thermal aging or PV cell efficiency decay over a 20-year lifespan.

\section{Scope for Future Research}
The foundation established in this thesis opens several promising vectors for future exploration:
\begin{enumerate}
    \item \textbf{Integration of Battery Energy Storage Systems (BESS):} Expanding the hybrid system to include a three port configuration with BESS would allow the DRL agent to learn energy arbitrage and primary frequency response, turning the intermittent hybrid system into a dispatchable, grid-forming asset.
    \item \textbf{Physics-Informed Reinforcement Learning (PIRL):} Future studies should investigate embedding the known differential equations of the DFIG into the neural network architecture or reward function. This could dramatically reduce the required training episodes and provide bounded safety guarantees during the exploration phase.
    \item \textbf{Multi-Agent Reinforcement Learning (MARL) for Wind Farms:} Scaling the unified control concept from a single turbine to farm-level optimization. A MARL framework could optimize wake effects between turbines while managing distributed solar PV generation throughout the network , coordinating local responses to achieve global grid stability.
\end{enumerate}
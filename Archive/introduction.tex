\\The rapid growth in global electricity demand, increasing penetration of renewable energy sources, and stringent environmental regulations have necessitated the development of sustainable and efficient power generation technologies. Wind and solar photovoltaic (PV) energy systems have emerged as dominant renewable sources due to their technological maturity, scalability, and declining cost trends. Among various wind energy conversion technologies, Doubly Fed Induction Generator (DFIG) based wind turbines are widely deployed because of their variable speed operation, reduced converter rating, and capability to independently control active and reactive power.

Solar photovoltaic systems complement wind energy generation by providing modular, clean, and reliable power, particularly during daytime operation. However, both wind and solar sources are inherently intermittent and stochastic in nature. When integrated into the power grid individually or collectively, they introduce significant challenges related to power fluctuation, voltage instability, reduced system inertia, and compliance with modern grid code requirements.

Hybridization of wind and solar energy sources offers a promising solution to mitigate intermittency and enhance power availability. A DFIG--Solar PV hybrid energy generation system can improve utilization of renewable resources and provide better operational flexibility. Nevertheless, the integration of multiple renewable sources through power electronic interfaces introduces nonlinear dynamics and strong coupling between electrical and mechanical subsystems, making system control increasingly complex.

Conventional control strategies employed in DFIG and PV systems are predominantly based on proportional--integral controllers implemented within vector control frameworks. Although these controllers are widely used in industry due to their simplicity, they require accurate system models and extensive parameter tuning. Their performance deteriorates under varying operating conditions, grid disturbances, and parameter uncertainties.

Recent advancements in artificial intelligence have introduced deep reinforcement learning as a powerful model-free control approach capable of handling nonlinear, multivariable, and uncertain systems. This thesis explores the application of advanced deep reinforcement learning algorithms for unified control of an integrated DFIG--Solar PV hybrid energy generation system.
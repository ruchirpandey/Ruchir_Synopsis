\section{Background and Motivation}

The global energy landscape is undergoing a fundamental transformation driven by the urgent need to mitigate climate change, decarbonize energy systems, and reduce reliance on fossil fuels. Renewable generation particularly wind and solar photovoltaic (PV) systems has become central to this transition. Recent assessments by international agencies project that renewable electricity generation will continue its strong growth trajectory, with solar PV and wind contributing the majority of global capacity expansion \cite{Uddin2023,Khare2023}. These technologies offer significant environmental and economic benefits but also introduce operational challenges arising from their inherent intermittency and reduced synchronous inertia.

Unlike conventional synchronous machines, which naturally provide rotational inertia and predictable electromechanical behaviour, renewable energy sources introduce high variability and limited controllability. Wind speed may change within seconds, while solar irradiance fluctuates due to cloud transients and diurnal patterns. Such variability contributes to operational uncertainty, complicating the tasks of maintaining frequency stability, voltage regulation and power quality challenges documented extensively in recent microgrid and renewable integration studies \cite{Ahmed2023,Shezan2023,Smith2022GridStability}. Recent comprehensive analyses confirm that the integration of inverter-based renewable energy sources leads to significant reduction in system inertia, increasing the rate of change of frequency (RoCoF) and degrading traditional frequency control efficacy \cite{Smahi2025,He2024}.

\textbf{Critical Challenge:} \textit{How can we design advanced control architectures that not only accommodate but optimally exploit the characteristics of hybrid, variable renewable energy sources while maintaining grid stability?}



\section{DFIG–Solar PV Hybrid Systems}

\subsection{System Architecture}

Integrating PV arrays directly into the DC link of Doubly Fed Induction Generator (DFIG)-based wind turbines has emerged as a promising hybrid system architecture. This configuration exploits the existing back-to-back converter hardware of DFIG systems, enabling the PV subsystem to share power electronic pathways. The resulting architecture has been highlighted in recent studies for its potential to reduce component count, lower costs, and enhance conversion efficiency \cite{Akhbari2023_DFIG,DFIGReview2019,Bhattacharyya2022}.

Recent work in 2025 has demonstrated significant advances in this topology. \cite{Prasad2025} presented a comprehensive DFIG-solar PV hybrid system using vector control techniques for both RSC and GSC converters, validated with real-time NREL solar irradiance data. 

\textbf{Advantages:}
\begin{enumerate}
    \item \textbf{Component reduction:} Avoids dedicated DC–DC and DC–AC conversion stages for the PV system
    \item \textbf{Cost savings:} Utilizes shared power electronic infrastructure, reducing capital expenditure by 15-25\% \cite{Prasad2025}
    \item \textbf{Improved reliability:} Fewer converters reduce thermal and switching stress points
    \item \textbf{Higher efficiency:} Minimizes cascading conversion losses, achieving overall system efficiency improvements of 3-5\% \cite{GridIntegratedDFIG2025}
   
\end{enumerate}

\textit{Critical Analysis:} While hardware simplification is advantageous, recent research shows that hybrid architectures may introduce additional dynamic interactions, requiring more sophisticated control strategies to fully leverage performance benefits \cite{Nguyen2023_DFIG,Tang2021_PVwind,Bhattacharyya2022}. The tightly coupled DC-link dynamics necessitate coordinated control approaches that can simultaneously manage wind power extraction, solar MPPT, and grid power exchange while maintaining voltage stability.

\subsection{Control Complexity}

DFIG–PV hybrid systems introduce significant control challenges due to the tightly coupled dynamics among three major subsystems. The Rotor-Side Converter (RSC) regulates rotor currents and facilitates maximum wind power extraction while managing electromagnetic torque. The Grid-Side Converter (GSC) manages DC-link voltage and governs grid power exchange including reactive power compensation. The Solar PV Array injects highly variable power into the shared DC link requiring continuous MPPT adaptation. A disturbance affecting any component propagates rapidly to the others due to the shared DC-link dynamics, creating a complex multi-input, multi-output (MIMO) control problem with strong nonlinear coupling. Contemporary analyses confirm that classical control approaches often struggle to simultaneously regulate DC-link voltage, optimize power extraction, and maintain grid compliance under such coupling \cite{Halivor2023_MPC,Joshal2023,GridIntegratedDFIG2025}. 

The control challenge is further compounded by multi-timescale dynamics where wind variations occur over seconds to minutes while solar fluctuations range from sub-second to minutes \cite{Prasad2025}, operational mode transitions requiring switching between sub-synchronous and super-synchronous operation \cite{Akhbari2023_DFIG}

These challenges have motivated increasing attention toward reinforcement learning and advanced predictive controllers as viable alternatives to conventional control architectures.

\subsection{Recent Advances in DFIG Control}

The field of DFIG control has witnessed significant developments in recent years, particularly in the application of advanced adaptive and intelligent control techniques:

\begin{enumerate}
    \item \textbf{Adaptive Filtering Approaches:} Recent work has demonstrated the superiority of logarithmic hyperbolic cosine adaptive filters (RZA-LHCAF) over traditional d-q control and LMS methods, achieving THD reductions from 9.34\% to 2.60\% \cite{GridIntegratedDFIG2025}
    
    \item \textbf{Integrated Energy Storage:} The integration of battery energy storage with DFIG-PV systems has shown promising results in power smoothing and grid support capabilities \cite{Bhattacharyya2022}
    
    \item \textbf{Hardware-in-the-Loop Validation:} Modern DFIG control strategies increasingly emphasize real-time validation using platforms such as OPAL-RT, ensuring practical deployability \cite{GridIntegratedDFIG2025,Prasad2025,Pandey2022PIICON}
\end{enumerate}

\section{Deep Reinforcement Learning for Power System Control}

\subsection{Evolution of Intelligent Control}

The application of artificial intelligence and machine learning to power system control has evolved rapidly over the past decade. Early approaches relied on rule-based fuzzy logic and traditional neural networks, which, while improving upon fixed-gain controllers, lacked the ability to learn optimal policies directly from system interactions \cite{Uddin2023}.

Deep reinforcement learning (DRL) represents a paradigm shift by enabling controllers to learn optimal decision-making policies through trial-and-error interaction with the system environment. Unlike supervised learning approaches that require labeled training data, DRL agents learn from reward signals that encode control objectives, making them particularly suitable for complex, nonlinear systems where analytical solutions are intractable.

\subsection{Recent DRL Applications in Wind Energy Systems}

The past two years have witnessed remarkable progress in applying DRL to wind energy conversion systems, with particular emphasis on the Twin Delayed Deep Deterministic Policy Gradient (TD3) algorithm:

\subsubsection{TD3 for Wind Turbine Control}

\cite{Zholtayev2024} presented the first comprehensive implementation of TD3 for both speed and current control loops in PMSG-based wind energy systems. Their work demonstrated several key advantages including superior robustness to parameter variations compared to model-based feedback linearization controllers, elimination of extensive parameter tuning required by conventional methods, and seamless handling of system nonlinearities without explicit modeling.

This pioneering study established TD3 as a viable alternative to traditional control methods for wind energy applications, opening new research directions for DFIG systems.




\subsubsection{Microgrid and Hybrid System Applications}

The application of TD3 has also expanded to microgrid control scenarios. \cite{Lee2023TD3} demonstrated dynamic droop control based on TD3 for AC microgrid systems, achieving optimized generation cost and frequency regulation. For hybrid renewable systems, \cite{WindStoragePINN2024} introduced physics-informed neural networks to accelerate the learning process in coordinated wind-storage control, incorporating power fluctuation differential equations directly into the DDPG architecture.

\subsection{DDPG vs. TD3: Algorithmic Advancements}

While Deep Deterministic Policy Gradient (DDPG) \cite{Lillicrap2015} pioneered continuous control with deep reinforcement learning, TD3 \cite{Fujimoto2018} addresses critical limitations:



Recent comparative studies in power systems consistently demonstrate TD3's superior performance, particularly in scenarios requiring tight voltage regulation and fast frequency response \cite{PhysicsConstrainedTD32024,Lee2023TD3,Liang2024}.

\subsection{Gap Analysis: DRL for DFIG-PV Hybrid Systems}

Despite significant progress in DRL applications for wind energy and microgrids, a critical research gap persists:

\textbf{Identified Gap:} While TD3 has been successfully applied to PMSG wind turbines \cite{Zholtayev2024}, DFIG virtual inertia control \cite{PhysicsConstrainedTD32024}, wind farm frequency regulation \cite{Liang2024}, and AC microgrid droop control \cite{Lee2023TD3}, no comprehensive study exists that applies TD3 to integrated DFIG-Solar PV hybrid systems with direct DC-link coupling.

This gap is significant because:
\begin{enumerate}
    \item DFIG-PV systems exhibit unique coupling dynamics not present in standalone configurations \cite{Prasad2025,GridIntegratedDFIG2025}
    \item The coordination of RSC, GSC, and PV MPPT poses control challenge not addressed by existing DRL literature
    \item Recent hybrid system studies rely on conventional PI or adaptive filtering approaches \cite{GridIntegratedDFIG2025,Bhattacharyya2022}, missing opportunities for DRL-based optimization
\end{enumerate}



